\documentclass[12pt]{article}

\usepackage[a4paper, total={6in, 9in}]{geometry}
\usepackage[english]{babel}
\usepackage{indentfirst}
\usepackage{fancyhdr}
\usepackage{setspace}
\usepackage{graphicx}
\usepackage{hyperref}
\usepackage{epsfig}
\usepackage{paralist}
\usepackage{lastpage}
% \usepackage[scaled]{helvet}
% \renewcommand\familydefault{\sfdefault}
\usepackage[T1]{fontenc}
\usepackage{url}
\usepackage{pgfplots}
\usepackage{tocbibind}
\usepackage{amsmath}
\usepackage{enumitem}
\usepackage{xcolor}

\pgfplotsset{compat=1.17}

\usepackage{xpatch}
\xpretocmd{\part}{\setcounter{section}{0}}{}{}

\pagestyle{fancy}
\fancyhf{}

\hypersetup{
  colorlinks=true,
  citecolor=blue,
  filecolor=black,
  linkcolor=blue,
  urlcolor=blue
}

\setlength{\parskip}{6pt}

\renewcommand{\headrulewidth}{0.3mm} % Top line width
\renewcommand{\footrulewidth}{0mm} % Bottom line width

\singlespacing

%Head
\lhead{\hspace*{0mm}\raisebox{3mm}{
  \epsfig{file=./logo/Aalto_en_1.pdf,height=11mm}}
}

\chead{\hspace*{50mm}\raisebox{7mm}{\hspace*{60mm}\small\begin{tabular}{r}
      \textbf{Programing Studio 2}\\CS-C2120\\\today\\
\end{tabular}}}

%Footer
\lfoot{}
\cfoot{}
\rfoot{\thepage}

\newcommand{\argmax}{\operatorname*{argmax}}
\newcommand{\argmin}{\operatorname*{argmin}}

\begin{document}

\begin{titlepage}
    \thispagestyle{fancy}
    \begin{center}
        \vspace*{1cm}
            
        \huge
        \textbf{General Plan}
            
        \vspace{0.5cm}
        \Large
        Balancer\\

        \normalsize
        \vspace{0.5cm}
        Final Project for CS-C2120
            
        \vspace{1.5cm}
            
        \textbf{Hau Phan}

        \normalsize
        886690
            
        \vfill

            
        Bachelor Degree in Science\\
        Data Science Program\\
        Years of study: 2020-2023
            
        \vspace{0.8cm}
            
        \normalsize
        Department of Computer Science\\
        Aalto University\\
        Finland, \today
    \end{center}
\end{titlepage}

\newpage

\tableofcontents
\addtocontents{toc}{\protect\hypertarget{toc}{}}
\newpage

\section{General Description}
A balance game where one tries to stack as many weights as possible to get the
most score. Player with the most score win the round. Player win the most round
win the entire game.
\subsection{Important Concepts}
\subsubsection{Weight}
A simple weight that have some mass. All the weights in the game have the same
mass of 1. Each weight yields more score the further they are from the center of
the scale.
\subsubsection{Scale}
A scale means a supported long arm on which you can place \textbf{other scales}
and \textbf{weights}:
\begin{verbatim}

                               x
                           x   x       x     x
                          -------    -------------
                          <1=B=1>   <3=2=1=C=1=2=3>
                             *             *
                             *           x *
                         ---------------------
                        <5=4=3=2=1=A=1=2=3=4=5>
                                   *
                                   *
               XXXXXXXXXXXXXXXXXXXXXXXXXXXXXXXXXXXXXXXXXXXXX
\end{verbatim}

More torque is applied to the arm if the weights are further from the middle
point. A small imbalance is permissible. The maximum allowed magnitude of the
imbalance is the same as if 1 weight was placed at the end of the scale before
the other weights were set. i.e. imbalance $\geq$ the ``radius” of the scale.
(See the example below).

Once a scale is flipped (becomes imbalance), it will be lost forever along with
the weights on it.
\newpage
\textbf{Example 1}
\begin{verbatim}

                   x
         x         x
        <3=2=1=A=1=2=3>
               *
               *
    XXXXXXXXXXXXXXXXXXXXXXXX
\end{verbatim}

If a weight is at a distance of 3 from the center of the arm, it pushes that
side down with the force of three weights. If the other side has two weights at
a distance of two, they push that side down with a force of four weights. (2 * 2
= 4) The imbalance is 3 - 4 = -1. The absolute value of the imbalance is less
than the radius of the scale, 3, so the scale is balanced.

\textbf{Example 2}

If the left side of the previous example did not have a weight, the scale would
have been unbalanced (flipped) since the imbalance would have been 4, which
exceeds the radius of the scale. In this situation, the scale is flipped and all
the weights is lost.  

You can also place other scales on the scale. The scale placed on the second
scale applies a force equal to the sum of the weights and scales on it.

\textbf{Example 3}

\begin{verbatim}

             x
         x   x       x     x
        -------    -------------
        <1=B=1>   <3=2=1=C=1=2=3>
           *             *
           *           x *
       ---------------------
      <5=4=3=2=1=A=1=2=3=4=5>
                 *
                 *
    XXXXXXXXXXXXXXXXXXXXXXXXXXXXXXXX
\end{verbatim}

Scale B is balanced, the right-side weighs 2 * 1 and the left one 1 * 1 which
does not exceed the radius of the scale. The scale B itself weighs three
weights.

Scale C is balanced since the left side weighs 1 * 2 = 2 and the right side 1 *
1 = 1. The difference of these is 1, which does not exceed the radius of the
balance. Scale C weighs 2 weights!

Scale A is also balanced. On the left side there is a scale B at a distance of
3, 3 * 3 = 9. On the right there is scale C at a distance of 4, 2 * 4 = 8, and a
weight at a distance 3. All together there is a weight of 11 weights on the
right side. The difference is 11 - 9 = 2, which is smaller than the radius of A,
so the scale is balanced.

\subsubsection{Players}
Players can be either \textbf{human} or \textbf{bot}. There can be many players
in a game/round (Usually 2-3 players).

\subsubsection{Capture and ownership}
\label{sec:cap}

Both the \textbf{weights} and \textbf{scales} during each round is either
\textbf{player-owned} or \textbf{wild} which are owned by no one. The wild
weights and scales are placed randomly at the beginning of each round and can be
captured by players. Each players can also captures each other weights and
scales. Captured scales give addition "buffs" for the owner's weights on the
scale.

Weights can be \textbf{stacked}. One can capture all the weights by placing
his/her weights on top of the stack.

To capture a scale, one must have \textbf{at least} $r$ weights more compared to
the player with the 2nd most weights on the scale, where $r$ is the radius of
the scale.  (See the example below)

\textbf{Example:}
\begin{verbatim}
                   a
             a b   a c
            <2=1=A=1=2>
                 1     
                 *     
                 *    
       XXXXXXXXXXXXXXXXXXXXX
\end{verbatim}

Here the scale is captured by player "a", since there are 3 "a" weights and only
1 "b" and "c" weight. The difference is $3 - 1 = 2 \geq 2$ the radius of the
scale.

In the beginning, all scales are wild and available to capture. Owner of a scale
will have the following benefits:
\begin{itemize}
  \item His/her weights on the scale can not be captured, even when other
    players place the weight on top. The moment the scales' owner change, all
    weight that should be captured will be updated according to the rules.
  \item Each of his/her weight will have its score multiplied by 2.
\end{itemize}

\subsection{Rules and Gameplay} 
\subsubsection{Gameplay}
\begin{itemize}
  \item There is a predefined number of round in one game. (Usually 5 for 2-3
    players)
  \item Each round will have a fixed number of weights that all players will
    draw from and place onto the scales. (Usually 8-12 weights per round)
  \item At the beginning of each round, one scale and 2-3 wild weights will be
    randomly placed by the computer with some predefined probability.
  \item The round starts and the players place the weights one at a time on the
    scales. The first player to place is the winner of last round.  It will be
    chosen at random on the first round.
  \item The round is over when there is no weight left. The score is then
    calculated and the player with the highest score win the round.
  \item The next round then starts and \textbf{continues} from the last round's
    state.
  \item The player who win the most round win the game.
\end{itemize}
\subsubsection{Scoring}

\textbf{Example:} 
\begin{verbatim}

            a
      a b   a
     <2=1=A=1=2>
          2          
          *         c b
          *   ?     c b
         <3=2=1=?=1=2=3>
                1
                *
                *
    XXXXXXXXXXXXXXXXXXXXXXXXXXXXXX
    Status: equalibrium 

    a: scale 2: (2 * 1 + 1 * 2) * 2 = 8 points 
       scale 1: 3 * 8 = 24 points
       -------------------------------
       total: 24 points

    b: scale 2: 1 * 1 = 1 points
       scale 1: 3 * 1 + 3 * 2 = 9 points
       -------------------------------
       total: 9 points

    c: scale 1: 2 * 2 = 4 points
       -------------------------------
       total: 4 points
\end{verbatim}

Here "?" represents wild weights and uncaptured scales. 

The scale with index 2 is capture by the player "a" so its points is multiplied
by 2

\subsection{Difficulty level}
Targeted level of difficulty: \textbf{Intermediate/Difficult}

The project first aims to meet all the requirement for Easy and Intermediate,
which require a text-based user interface and a graphical one. To meet the
Difficult requirements, the project also implements an intelligent computer
opponents and extends existing rules by introducing the capturing and wild
weights mechanics. 

Additional features: \textbf{Online Multiplayer/RL Bot}

These are the two additional features that I am personally really interested in
implementing and learning more about. However, it is not guaranteed that these
functionality will be fully realized at the final deadline. 

\section{User Interface}
\subsection{Text based}
In each turn, the program will print out the current state of the game in a
format similar to the example format below. The program will print out the round
number, the current score of each players and the state of each scale. Then it
will ask for user inputs and place the weight accordingly. An example output
would be as follow:
\begin{verbatim}
#################### ROUND 1 ####################
__________________________________________________
| Score board                                    |
|________________________________________________|
| a: 27 points                            human  |
| b: 18 points                            bot    |
| c: 15 points                            bot    |
|________________________________________________|

Game State:

Scale [a]
-------------------------------
<a, 6>: <<d,3>--[a]<b,1>-|b,b|> 
TORQUE: [9--3--7] 
STATUS: Balanced
OWNER : None


Scale [b]
-------------------------------
<b, 1>: <--[b]-<c,1>> 
TORQUE: [0--2--2] 
STATUS: Balanced
OWNER : None


Scale [c]
-------------------------------
<c, 1>: <|b|---[c]----> 
TORQUE: [4--4--0] 
STATUS: Balanced
OWNER : None


Scale [d]
-------------------------------
<d, 3>: <|c,c|---[d]---|a|> 
WEIGHT: [8--4--4] 
STATUS: Balanced
OWNER : None

>>>>>>>>>>>>>>>>> A(HUMAN) TURN <<<<<<<<<<<<<<<<<

Which scale ? (a,b,c,d): 3
Which arm ? (left/right): right
How far from the center ? (1-4): 4
\end{verbatim}

In the example output, the scoreboard and the user input section is quite
straightforward. The game state is just a listing of each scale's state, in the
order of the index. For example, consider the scale with index 1, which is also
the scale being placed on the ground:
\begin{verbatim}
Scale [a]
-------------------------------
<a, 6>: <<d,3>--[a]<b,1>-|b,b|> 
TORQUE: [9--3--7] 
STATUS: Balanced
OWNER : None
\end{verbatim}

Here <a,6> is its code ("a") and its weight ("6"). A scale is thus represented
in the form:
\begin{align*}
  <\text{scale's code}, \text{scale's weight}>
\end{align*}

On the parent scale, child scales are also represented is this way: <d,3> is
scale "d" with 3 weight on it, <b,1> is scale "b" with 1 weight on it. 

$[scale's code]$ i.e [a], [b],... represents the center of the scale and each hyphen (-) denotes
an empty space on the scale with no weights.

A stack of weights is represented by two vertical bars surrounding the weights.
Each weight is represented by its respective owners' letter: i.e "a", "b", "c",
... The bottom of the stack is equivalent to the left vertical bar and the top
of stack corresponds to the right vertical bar.

"TORQUE:" contains three numbers (ex: [9--6--7]), from left to right are: the
left torque, the radius of the scale and the right torque. If the difference
between the left and right number is greater than the middle number, the scale
is imbalance.

"STATUS:" and "OWNER:" indicate whether the scale is balance or not and the
owner of the scale respectively (see \autoref{sec:cap})

\subsection{Graphical User Interface} 
The user can also interact with the program through a graphical user interface.
There are three main components in the main UI: the game screen, the scoreboard and
the form. 
\begin{itemize}
  \item Game screen: display the current state of the game which includes
    where the scales are located, how the weights are placed, who currently owns
    which scales (by color),...
  \item Scoreboard: display all the relevant information for the user to
    make a decision on where to place his/her weight, which might include the
    current score, the number of weights and scales of each players, the number
    of weights left...
  \item Form: where the users can fill in the information of where the
    weight should be placed. There is also a submit button to confirm the
    placement of the weight.
\end{itemize}

\begin{center}
  \includegraphics[width=1.2\textwidth]{Graphical User Interface.png}
\end{center}

\section{Files and file formats}
The state of the game can be saved in semi-readable text format as follow:
\begin{verbatim}
BALANCER 1.3 SAVE FILE

# Meta
Number of Player: 3
Human: a
Bot: b,c
Round: 2
Turn: c

# Scale
_,0,5   : -4,c,c | 2,b
a,3,4   : -1,b |   2,a,c
a,-1,4  : 1,a    |-1,a

END

\end{verbatim}

\begin{itemize}
  \item The first line of the file must be of the form: 
    \begin{align*}
      \text{BALANCER (VERSION) SAVE FILE}
    \end{align*}
  \item Each block of data has a header and a body. The first character of the
    header must be the "\#" symbol to be identified as a header. The header
    marks the start of the block.
  \item Each row in the body contains information about the state of the game
    and is of the form:
    \begin{align*}
      \text{IDENTIFIER : DATA}
    \end{align*}
    \item The IDENTIFIER and DATA is different for different block, here there
      are only two different type of block: \textbf{Meta} and \textbf{Scale}.
  \item The Meta block contains metadata of the game. There are 6 main
    identifier in the Meta block, which is shown in the example above. More
    identifier might be added in the course of development. 
  \item The Scale block contains the position of each scales and weights when
    the game is saved. The IDENTIFIER is the scale's position and the DATA is
    the position of each weight on the scale:
  \item Noted: In the following example file, negative indices indicate the
    object is on the left arm of the whichever scale it is on, which we will
    call the \textbf{parent scale}.  Positive indices indicate the object is on
    the right arm.
  \item For example, this line in the Scale block:
    \begin{verbatim}
                        1,3,3  : -1,b  |  2,a,c
    \end{verbatim}
    from left to right indicates: the parent scale (scale with index 1), where
    on the parent scale this scale is placed (on the right arm, distance 3 from
    the center) and the radius (the scale have radius 3)
  \item After the colon are the weights' position on the scale, it must be
    listed from left to right (thus the indices must be from smallest to
    biggest). 
    \begin{itemize}
      \item (-1,b) indicates on the left arm, distance 1 from the center, there
        is a weight belong to the $b$ player
      \item (2,a,c) indicates on the right arm, distance 2 from the center,
        there are weights stacked on each other. The bottom weight belong to
        player $a$ and the top weight belong to player $c$.
    \end{itemize}
  \item The END keyword represents the end of the saved file. All line after
    this keyword will not be interpreted by the parser.
  \item Noted: Any excess white space will be ignored by the parser.
\end{itemize}
\end{document}
